\documentclass[10pt]{article}

\begin{document}
\author{Klev}
\title{Computing Arbitrary Functions of Encrypted Data}
\maketitle

\begin{itemize}
	\item Fully homomorphic encryption scheme.
	\item Keep data private, only people with access can decrypt it.
	\item Still, possible to compute on the encrypted data.
	\item Cloud computing compatible with privacy
\end{itemize}

- Fully homomorphic encryption

- Fully: No restrictions on the operations that can be performed

- Send description of function $f$ to the server. Server computes the function on the encrypted data. Returns encrypted data. User decrypts.

- Useful whenever the response can be encrypted.

- Alice and her jewelry store.\\

\section{Homomorphic Encryption: Funcitonality}

Three algorithms: $KeyGen$, $Encrypt$, $Decrypt$


Symmetric encryption scheme: Key used for both encryption and decryption

- Asymmetric encryption scheme: Public key for encryption, Private key for decryption

- HE can be either symmetric or asymmetric. Focus on asymmetric.

- Fourth algorithm: $Evaluate$. A set of permitted functions $F_e$. $Evaluate$ can handle functions in $F_e$. Others are undefined.\\

\section{Requirements}

- Require that decrypting $c$ (output from Evaluate) takes the same amount of time as decrypting $c_1$ (output from Encrypt).

- $c$ is the same size as $c_1$

- "Compact ciphertext requirements"

- Size of $c$ and the time required to decrypt it is independent of $f$

- Fully homomorphic if it can handle all functions, has compact ciphertexts and $Evaluate$ is efficient.

- The work required by Alice to extract jewelry has no connection to the time spent by workers to assemble the piece.

- Measure complexity of function $f$: Running time $T_f$ and size $S_f$ of a boolean circuit. 

- Break down the computation of $f$ into AND, OR and NOT gates.

- Obtain fully homomorphic encryption by operating on ciphertexts using add, subtract and multiply.

- If encrypted, random-access can not speed up the work. $Evaluate$ must touch all points. Runtime is at least linear. 

- Not touching them would leak information, saying that they are irrelevant for function $f$.

- Trade off: the data is contained in 1\% of my area. Reduce computation by factor 100.

- Should be able to specify the amount of data received from query. Unavoidable with padding and truncating.



\section{Homomorphic Encryption: Security}

- Semantic security against chosen-plaintext attacks (CPA)

- Given $c$ and two messages $m_1$ and $m_2$, it is \emph{hard} for an adversary to determine which plaintext the ciphertext decrypts to.

- "Hard": Guesses correctly with probability 1/2 + $e$ = Advantage

- Deterministic: Same plaintext encrypts to same ciphertext every time.

- The scheme must be probabilistic; a plaintext encrypts to multiple ciphertexts.

- Malleability weakens deterministic schemes, not semantically secure.


\section{A Somewhat Homomorphic Encryption Scheme}

- First as a symmetric encryption scheme.

- $N = \lambda$, $P = \lambda^{2}$, $Q = \lambda^{5}$

- KeyGen: Key is random P-bit odd integer p

- Encrypt(e,m): $m'$ = random N-bit number such that $m'$ = $m$ mod 2. $c \leftarrow m' + pq$

- Decrypt(p,c): $(c mod p) mod 2$, where $c mod p$ is the integer $c'$ in $(-p/2, p/2)$ such that $p$ divides $c-c'$.

- $c mod p$ is the noise associated to a ciphertext. Distance to nearest multiple of $p$.

- Consider $Mult_e$: $c = c_1 * c_2$. Noise to $c_i$ = $m_i'$. We have that $c = m'_1 * m'_2 + pq'$

- Multiplication tends to increase the noise faster than addition and subtraction.

- $c_1$ and $c_2$ with $k_1$ and $k_2$-bit noises = roughly ($k_1 + k_2$)-bit noises.

- The functions that can be handled are those where $|f^{\dagger}(a_1,...,a_t)|$ is always less than $p/2$.

\section{Bootstrappable Encryption}

- Work on box 1, put it inside box 2 which contains key 1, open box 1 and continue work through box 2.

- Somewhat homomorphic encryption: Supports Add, Subtract, Multiply, and a limited set of operations, until the noise gets too large.

- Key for box $i$ represents an \emph{encrypted secret decryption key}.

- Most important function: The decryption function (Open up the inner box to continue work)

- Decrypts itself $\rightarrow$ bootstrappable

\subsection{Boostrappable to Fully Homomorphic}

- Suppose $e$ can handle decryption, ADD, SUB and MULT.

- If these four algorithms are in $f_e$, we can construct an encryption scheme that is fully homomorphic.

- $Recrypt_e(pk_2,D_e, sk_1, c_1)$ Recrypt outputs an encryption of $m$, but under the new key $pk_2$. The message is encrypted twice under different keys $pk_1$ and $pk_2$.

- Can "peel off" the layers like an onion, but the $evaluate$ algorithm "opens" the inner encryption. (Like in the box example).

- Decryption the inner layer removes noise. Using $Evaluate$ with $pk_2$ adds new noise. If the added noise is less than what we remove in the decryption, we've made progress.

- Public key consists of a sequence of public keys $(pk_1, pk_2, ..., pk_n$ encrypted under secret keys $sk_{i+1}$

\subsection{Circular Security}

- Safe to reveal the encryption of a secret key $s_k$ under its own associated public key $p_k$.

\subsection{Greasing the Decryption Circuit}

- $m \leftarrow (c mod p) mod 2$ == $m \leftarrow LSB(c) \oplus LSB(\lfloor c/p\rceil)$

- If not $\lfloor c/p\rceil$ is complicated, then $e$ is bootstrappable and FHE is possible. Not possible with the current scheme. Let's create $e*$

- Replace $e$'s decryption function, which multiples two long numbers, with a decryption function that adds a fairly small set of numbers.

\subsubsection{Transformation}

- KeyGen: Run the algorithm to obtain $p_k$ and $s_k$($s_k$ is an odd integer $p$.

- Generate a subset from $s_k$ of rational numbers that sums up to roughly $1/p$. $s_k$ will then be the subset $y$ of the numbers encoded as a vector. $p_k* \leftarrow (pk, y)$

- This is a "hint" about the secret key. Adds the "grease" to the system. Revealing this obviously impacts security.

- Encrypt: Run Encrypt($p_k$,m) to obtain ciphertext $c$. For $i \in \{1, ..., \beta\}$, set $z_i \leftarrow c * y_i  mod 2$. The ciphertext $c*$ is now $c$ and $z = \{z1, ... ,z_{\beta}\}$

- The hint is used to postprocess the ciphertext. The idea is to leave less work for the Decrypt algorithm.

- Used in 'Server-aided cryptography'.

- The hint is statistically dependent on $s_k$, but enough for the server to decrypt efficiently on its own.

- In our case: The encrypter or evaluator plays the role as server

- Decrypt: Output $LSB(c) \oplus LSB(\lfloor \sum s_iz_i\rceil)$

- Works because $\sum s_iz_i = \sum c * s_iy_i = c/p mod 2$

- Important: Replace the multiplication of $c$ and $1/p$ with a summation that contains only $\alpha$ nonzero terms.

- Add: ($p_k*, c_1*, c_2*$). Extract $c_1$ and $c_2$ from $c_1*$ and $c_2*$. Run $c \leftarrow Add(p_k, c_1, c_2)$. The result consist of ciphertext $c*$. $c* = c$ and the result of postprocessing $c$ with $y$.

\subsection{How to Add Numbers}

- Let us consider the computation of $\sum s_iz_i$.

- 

\subsection{Security of the Transformed Scheme}

- The encryption key contains a hint about the secret $p$.

- Semantically secure as long as the GCD problem and the SSSP is hard. Which it is, as long as $\alpha$ and $\beta$ are set appropriately.

\subsection{Conclusion}

- FHE is possible

- Currently, all known FHE follows the pattern. Bootstrappable homomorphic encryption scheme with an evaluate function on the decryption function.

- Computationally expensive, Decryption function as a circuit, and an $Evaluate$ that replaces the bits in the circuit with ciphertexts.

- The decryption function is highly parallelizable - may make an enormous difference. 
\end{document}