\chapter{Introduction}
\label{chp:introduction}

\section{Motivation}

Cloud services are becoming larger and more complex. Users want their content available wherever they are, forcing applications to store content in the cloud. Companies such as Apple \cite{apple_health}, Microsoft \cite{microsoft_health} and other corporations are increasing their focus on health information and how your personal information can be integrated into their services. Medical research facilities stores tremendous amounts of personal data, and are currently looking into how to share their research material across facilities and borders. Competitions, such as iDASH \cite{iDASH} have been held in order to create the best suited methodology for how to securely store sensitive information. But in 2015, encrypting and storing sensitive information is not enough. Sensitive information is valuable, and even more valuable if we are able to utilize it in a secure manner. 
%Along with these types of sensitive data, follows great responsibility and security measures. So when developing applications and systems, data security and confidentiality are important topics.
 

As developers, we are left with two choices when creating applications utilizing some sort of data storage. Option one is to build our own server farm or data center on a secure site, which is a rather expensive solution with costs related to both hardware, maintenance, electricity and rent. Option two is to out-source the storage to a cloud provider, where the developer stumbles into another problem. How can we guarantee that the data is stored securely, given that we cannot necessary trust the provider?

The first solution that comes to our mind is that the user could encrypt its data with a strong block cipher, and then decrypt the result at the client side. Sure, but this does not really solve anything. Problems arise when the application needs to perform operations that are too heavy for an ordinary client's machine. What if there existed a database system that could solve these problems for us? A system where the data is safely stored in the cloud without the possibility of having database administrators snooping around, or adversaries able to extract any information in the (un)likely case of a database breach. A system that could perform all kinds of operations on our data and send the encrypted result back, without leaking any sort of information about the query, the result, the data itself or any intermediate values. Homomorphic encryption schemes might be our rescue.

In short terms, \gls{he} is a cryptographic property describing the ability to perform certain operations on encrypted data (ciphertexts) without decrypting it first. \gls{fhe} is the enhanced version where the encryption scheme is capable of performing all functions efficiently \cite{Gentry_thesis}. While still in research mode, today's fully homomorphic encryption schemes' biggest challenge is the fact that they are highly inefficient. While being a hot research topic, we may assume that \gls{fhe} schemes still have a long way to go before being deployed in commercial systems \cite{naehrig2011can}. However, if the practical systems ever reaches the theoretic potentiality of \gls{fhe}, such schemes and systems could be a possible game changer for cloud based services in the future.

In 2011, a research team at the \gls{mit} presented CryptDB, which is an encrypted database system providing practical and provable confidentiality \cite{CryptDB_Main_Paper}. CryptDB allows its users to interact with the system by issuing \gls{sql} queries over the encrypted data in an efficient manner. While being a proof of concept, its pioneering design is being used by large companies such as Google and Microsoft \cite{cryptdb_homepage}.

%\noindent
%???\\
%Maybe we can achieve FHE in another way?\\
%For now: What is practically possible?\\
%???
\section{Problem}

The objective of this project is to critical assess CryptDB in terms of functionality and security, along with developing a small demonstration application to better understand its possible use cases. This project has resulted in an overview of the CryptDB system, a guide on how to install and run CryptDB regardless of operating system, and finally a small sample application for demonstration purposes.

\section{Overview of the Report}

In Chapter \ref{chp:background}, a general background on databases and homomorphic encryption will be presented. Chapter \ref{chp:overview_cryptDB} consists of an analysis of CryptDB in terms of functionality and security, while Chapter \ref{chp:attacks} covers some of the different attacks on CryptDB that has been proposed in the past years. The demonstration application and its results are presented in Chapter \ref{chp:software}. Finally, Chapter \ref{chp:conclusion} contains a comparison of CryptDB to a \gls{fhe} scheme, as well as a discussion on the future of CryptDB as a possible solution for \gls{he} schemes.