\chapter{Conclusion}
\label{chp:conclusion}

\section{Impact of CryptDB}

CryptDB is the first practical implementation of a somewhat homomorphic encryption scheme, and has shown a possible way to go when creating encrypted databases. A lot of well-known companies such as Google, Microsoft and SAP, as well as a lot of start-ups, are using CrytpDB's building blocks and putting its ideas into their commercial software \cite{cryptdb_homepage}. As CryptDB is enabled exclusively for single-user applications in the nearest future, it is difficult to point out useful applications involving highly sensitive data. However, CryptDB seems as a well-suited approach for creating diaries, contact lists, and other applications targeted for a single user to use.

\section{Conclusion}

This project has been about exploring CryptDB's main features and how it works. The results have been an overview over the system and its way of utilizing different types of encryption in order to enable a somewhat homomorphic encryption scheme. CryptDB provides a neat software, which is easy to interact with after the installation and set-up has been successful. It comes with a brief "how to install and run" guide available in their GitHub repository, however, this guide is a bit vague. Therefore, some of the attention of this project has been given to create a small and easy guide for future projects wishing to explore CryptDB to follow.

Along with the exploration, there has also been created a small application for testing the various queries that CryptDB allows and to observe the behaviour of the proxy server. When seen in connection with the attacks presented in Chapter \ref{chp:attacks}, CryptDB certainly provides an interesting approach towards a practical homomorphic encryption scheme, but there may still be some work to be done.


