\pagestyle{empty}
\begin{abstract}

\noindent
CryptDB is a practical homomorphic encryption scheme for SQL queries, enabling storage and processing of encrypted data. The system consists of two entities. One being a custom proxy server, which responsibility is to intercept and rewrite queries that are sent between the user and the database server, which is the second entity. The database server is a regular relational database supporting SQL queries, where a small set of CryptDB specific functions have been implemented.

CryptDB utilizes an SQL-aware encryption scheme, meaning that data items that are sent to the database are encrypted multiple times with different encryption schemes. The encryption schemes are encapsulated in each other from most secure to least secure. By doing so, CryptDB provides confidentiality to the data, while still being able to execute most queries seen in regular SQL traces. 

This report assesses the different encryption schemes used in CryptDB along with its security and limitations. The report also indicates that some of the proposed building blocks suggested in CryptDB for homomorphic encryption schemes may not be that well suited for real world applications. It also describes how to install and set up CryptDB, along with a demonstration application to assess and present some of its use-cases.  

\noindent
\textbf{Keywords:} CryptDB, Databases, Homomorphic Encryption, SQL
\end{abstract}